
%%%%%%%%%%%%%%%%%%%%%%% file typeinst.tex %%%%%%%%%%%%%%%%%%%%%%%%%
%
% This is the LaTeX source for the instructions to authors using
% the LaTeX document class 'llncs.cls' for contributions to
% the Lecture Notes in Computer Sciences series.
% http://www.springer.com/lncs       Springer Heidelberg 2006/05/04
%
% It may be used as a template for your own input - copy it
% to a new file with a new name and use it as the basis
% for your article.
%
% NB: the document class 'llncs' has its own and detailed documentation, see
% ftp://ftp.springer.de/data/pubftp/pub/tex/latex/llncs/latex2e/llncsdoc.pdf
%
%%%%%%%%%%%%%%%%%%%%%%%%%%%%%%%%%%%%%%%%%%%%%%%%%%%%%%%%%%%%%%%%%%%


\documentclass[runningheads,a4paper]{llncs}

\usepackage{amssymb}
\setcounter{tocdepth}{3}
\usepackage{graphicx}

\usepackage{url}
\urldef{\mailsa}\path|{jrcole,ying.xu,reitter}@psu.edu|    
\newcommand{\keywords}[1]{\par\addvspace\baselineskip
\noindent\keywordname\enspace\ignorespaces#1}

\begin{document}

\mainmatter  % start of an individual contribution

% first the title is needed
\title{How People Talk about Armed Conflicts}

% a short form should be given in case it is too long for the running head
%\titlerunning{How People Talk about Armed Conflicts}

% the name(s) of the author(s) follow(s) next
%
% NB: Chinese authors should write their first names(s) in front of
% their surnames. This ensures that the names appear correctly in
% the running heads and the author index.
%
\author{Jeremy Cole
\and Ying Xu\and David Reitter}
%
%\authorrunning{Jeremy Cole, Ying Xu, and David Reitter}
% (feature abused for this document to repeat the title also on left hand pages)

% the affiliations are given next; don't give your e-mail address
% unless you accept that it will be published
\institute{The Pennsylvania State University \\
College of Information Science and Technology\\ 
University Park, Pennsylvania, USA\\
\mailsa\\
%\mailsb\\
%\mailsc\\
}

%
% NB: a more complex sample for affiliations and the mapping to the
% corresponding authors can be found in the file "llncs.dem"
% (search for the string "\mainmatter" where a contribution starts).
% "llncs.dem" accompanies the document class "llncs.cls".
%

%\toctitle{How People Talk about Armed Conflicts}
%\tocauthor{Jeremy Cole, Ying Xu, and David Reitter}
\maketitle


\begin{abstract}
Armed conflicts around the world produce displacement, injury, and death. 
This study examines how anonymous and quasi-anonymous internet commenters discuss such conflicts.  Specifically, we ask how permissible it is to express positive or negative sentiments about these conflicts as a function of variables including region, conflict nature, and severity. Data from the Armed Conflicts Database is aggregated to identify a number of potential factors that may influence views on acceptable sentiments.  A large-scale sample of the Reddit comment forum was coded for positive and negative sentiments using sentiment analysis techniques. Permissibility is judged using the Reddit voting features. Regressions reveal that positive sentiments are found not permissible depending on the number of fatalities, and that negative sentiments are found permissible for certain regions, more permissible for older conflicts, and more permissible for territorial conflicts.  A number of alternative, non-correlated predictors is presented.

% These results can help disentangle factors such as a conflict’s location from its intensity, which could help explain how people view the multitude of conflicts around the world.
\keywords{Behavioral and Social Sciences; Corpus Linguistics; GLMM; Armed Conflicts; Public Opinion}
\end{abstract}

\section{Introduction}
According to Armed Conflict Database, there are 42 active conflicts around the world causing 180,000 fatalities and more than 12 million refugees. For those who live in relatively peaceful area, such as North America, their perception, opinion and action towards these international crises are important. Collectively, these perceptions are influential in shaping their countries' foreign policy \cite{Gelpi2009}. 

Traditional media and social media interact to help form these perceptions. Conversely, consumers of media implicitly or explicitly rate the quality of the content. Thus, contributors' perception of events forms something of an exacerbatory cycle, where biases can potentially further biases. The question is what are those biases to begin with?

We are motivated to understand how people talk about and act upon armed conflicts. Specifically, we want to understand when and why people accept negative discussions and positive discussions. We examine the effect of features like the number of fatalities, refugees and Internally Displaced People (IDP),  along with a variety of features that may influence objectivity, such as the location of the conflict. We hope to discover relations between these features and public opinions of specific armed conflicts.

%Reddit's homogenaeity
%Tie to Journalism principles
\section{Related Work}
\subsection{Public Opinion of Armed Conflicts}
We investigated past research on the perception of armed conflicts from the perspectives of both the decision makers and the general public. Cognitive consistency and distortion, and their implications in perceiving international politics were examined with decision makers \cite{Jervis1976}. Public opinion on armed conflicts from World War II to Vietnam, the Gulf, Afganistan and Iraq war have also been investigated to find patterns of public responses to international conflicts. Particularly, with the pervasive access of the Internet and use of social media, public opinions are becoming more influential on political decision-making and even changing political cycle \cite{Shirky2011}. However, systematic studies of public opinion from social media on armed conflicts is limited. Studies focusing on social media in armed conflicts often regard social media tools as agents or platforms to frame views and coordinate actions: for example, arranging protests and coordinating uprisings \cite{Lim2012}.

Public perception of armed conflicts through both traditional and social media suffers from several limitations. Advanced communication technologies make it possible to disseminate information about various conflicts around the world \cite{Sacco2015}. However, government and news organizations cover and frame conflicts in a certain way that adds their own bias. For example, selection of foreign events to report largely goes to war, terrorism and political violence \cite{Nossek2004}. Even if it is unintentional, this causes a significant limitation for the public to be able to obtain the broad-spectrum of information they need to evaluate such conflicts objectively \cite{Nelson1997}. Furthermore, news agencies generally focus on news in their own country or nearby countries, shrinking the amount of time focused on international coverage \cite{Seib2004}.

Despite these biases, growing studies are covering the relationship between the characteristics of armed conflicts and the public perception of them. Berinsky has found that public reactions to conflicts has been shaped less by their defining characteristics, such as fatalities and resources costs, than by the same political interests and group affiliations that influence ideas about domestic issues \cite{Berinsky2009}. Gartner at el. investigated this idea more deeply; he found that marginal fatalities are important in explaining opinion than cumulative total number of fatalities \cite{Gartner1998}. They also studied the relationship between race and opinion towards the Vietname war, and found that Asians from California supported the second part of Vietnam war given the anti-communist nature \cite{Gartner2000}. 

\subsection{Social Computing}
The usage of large-scale datasets from social media have produced a flood of research and discourse in areas like sociology, political science, and psychology. The roles of social media as a platform in coordinating activities pre-conflict, during the conflict and post-conflict have been studied extensively. For example, time-series analysis of big social media datasets offers evolving theories of public opinion and public attention on economic and social welfare, foreign affairs, and environmental issues \cite{RussellNeuman2014}. Twitter, in the realm of social media, is one of the most popular corpora for studying armed conflicts and other political events. 

Few studies have examined public opinions of multiple armed conflicts together by using social media data, especially the relaitonship between charateristics of the conflicts and the level of acceptability of public opinions.

\subsection{Reddit Corpus}
However, Twitter has a couple of drawbacks. The first is that there is no clear \textit{anchor}, or shared context, for what people are discussing. Additionally, there is no way to curate content. Reddit, a large Internet forum, also has the potential to explore users' perception and action on various issues. 

On Reddit, people start discussions by submitting either a hyperlink or text post. On many discussion Reddits, this submission is generally a link to a news article. Then, people can reply to these submissions. Users can also reply to the replies to these submissions (and so on). Additionally, users can upvote or downvote both submissions and individual comments. 

In the case of submissions, highly upvoted submissions are more likely to be seen, while highly downvoted submissions are less likely to be seen. Under an individual submission, all comment replies to that submission are sorted by score, and all replies to those comments are sorted by score, and so on. If a comment is sufficiently downvoted, it will be hidden. Hidden comments can still be replied to, upvoted, and downvoted. When a comment receives upvotes and downvotes, it is considered \textit{controversial}.  

Reddit is organized into an infinitely expandable number of smaller forums, normally referred to as subreddits. Each of these subreddits can be about any given topic, general or specific. For instance, it's possible to have a subreddit about science, biology, and genetics. These three subreddits would (or at least could) be completely independent, and they are organized in a flat rather than hierarchical fashion.

Besides the organizational ways that Reddit is different than Twitter, it also forms a subculture. The vast majority of the discussions are in English. Over half of the users are in the United States, and the majority are fairly young \cite{pewinternet}. 

Researchers normally investigate phenomena in a small selection of subreddits. For instance, Hurricane Sandy was examined with Reddit's /r/sandy subreddit to understand how types of networked gatekeeping impact the framing of a crisis situation; Reddit's voting system forms a non-traditional gatekeeper for what kind of information becomes negotiated as relevant \cite{Leavitt}. A different study examined the sharing and seeking of mental health information to examine factors that drive social support \cite{dechoudhury2014mental}.

\subsection{Armed Conflict Database}
The Armed Conflicts Database was developed by International Institute for Strategic Study, containing various indexes of armed conflicts around the world \cite{conflictDB}. They group conflicts into several regions, including Caribbean and the Americas, East Asia and Australasia, Europe, Middle East and North Africa, Russia and Eurasia, South Asia, and Sub-Saharan Africa. The Armed Conflict Database contains data on the total number of fatalities from a conflict, the year the conflict started, the current IDP, the current number of refugees, the new refugees and fatalities every year, and the IDP every year. It additionally contains a variety of factors that relate to the conflict's origin, such as ethnic violence. 





\section{Research Questions}

\begin{enumerate}
\item{ How acceptable is it to talk negatively or positively about conflicts in various regions?} 
\item { How acceptable is it to talk negatively or positively about conflicts of various origin?} 
\item { How much do the externalities of these conflicts, such as the number of fatalities or refugees, play a role? }
\end{enumerate}


%\section{Reddit Corpus}
To answer our research questions, we chose to use the Reddit corpus. Reddit is a large online internet forum where users are able to discuss a wide array of  topics. To start a discussion, someone submits either a hyperlink or text post. Then, people can reply to these submissions. Users can also reply to the replies to these submissions (and so on). Additionally, users can upvote or downvote both submissions and individual comments. 

In the case of submissions, highly upvoted submissions are more likely to be seen, while highly downvoted submissions are less likely to be seen. Under an individual submission, all comment replies to that submission are sorted by score, and all replies to those comments are sorted by score, and so on. If a comment is sufficiently downvoted, it will be hidden. Hidden comments can still be replied to, upvoted, and downvoted. When a comment receives upvotes and downvotes, it is considered \textit{controversial}.  

Reddit is organized into an infinitely expandable number of smaller forums, normally referred to as subreddits. Each of these subreddits can be about any given topic, general or specific. For instance, it's possible to have a subreddit about science, biology, and genetics. These three subreddits would (or at least could) be completely independent, and they are organized in a flat rather than fashion.

Besides the organizational ways that Reddit is different than Twitter, it also forms something more of a subculture. The vast majority of the discussions (and all discussions we examined) are in English. Over half of the users are in the United States, and the majority are fairly young. 

Our data set consists of approximately two terabytes of Reddit data, ranging from the year 2011 to the year 2014. These comments span every subreddit. 


\section{Methods}
\subsection{Data Preparation}
We first gathered a list of comments by checking if they contained any keywords associated with that conflict. The goal of this list was to be broad enough to capture any given comment while also ensuring that comments would be a positive identification of the topic. It is a near guarantee that some comments that were collected would not actually pertain to the given conflict. However, with enough data, we hoped to filter out the noise. Our answer to this is to use enough data to filter out most of the noise. The biggest source of comments was /r/worldnews, with many of the others coming from similar subreddits, implying that most of the comments were likely about the intended topic. The comments came from over a thousand subreddits; some of these could be falsely identified to be relevant, while some of them could be off topic. 

As discussed earlier, we ran sentiment analysis on the comments we collected. Interestingly, there were many more negative comments than positive comments. It is possible that this is due to the subject matter; in general, pity and sadness might be more common responses to tragedy than optimism and hope. It could also be due to the nature of the corpus itself: young people on pseudononymous Internet communities could be more negative than average. Nonetheless, we separated the negative and positive comments and pruned the neutral ones.

Lastly, we must define what it means to be acceptable. In Reddit, this is  simply whether a comment has been upvoted or downvoted more. All comments start with a score of one; the idea is that you always approve of your own comment. As comments with a score of one thus contain no useful information about how the community perceived them, they were excluded. 

As Reddit manipulates how probable it is for people to see a post, we did not consider the actual number of upvotes or downvotes received to be useful information. We thus coded anything upvoted past one as not unacceptable, and anything downvoted below one as unacceptable.

Posts that were deemed controversial by Reddit were vanishingly rare, making up less than 2\% of the filtered comments. Thus, people clearly tend to downvote comments that were already downvoted and upvote comments that were already upvoted (or to take no action at all). Therefore, a binary value of approval seemed more appropriate than any other choice.

Our two filters resulted in 781 positive comments and 14289 negative comments. The substantial difference in sentiment furthered our desire to split the analysis. Comparing these two models could help us determine if there are different circumstances when expressing different types of sentiment is more acceptable. Due to the overwhelming number of negative comments (over 95\%), it is difficult to interpret the negative sentiment as much more than the norm. If this is the case, then our results, instead of reflecting when negative sentiment would be acceptable, would reflect when posters are more likely to post something that is unacceptable. We'll discuss what that might mean later.

\subsection{Model}
We use a Logistic Linear Mixed Effects Model (GLMM) to attempt to explain which sentiments were acceptable as determined by the conflict.

\subsubsection{Response}
Every comment is coded as either unacceptable (1) or acceptable (0). This means that for some feature of a conflict that a comment is about, positive $\beta$ implies that comment is more likely to be disapproved of. Conversely, negative $\beta$ implies that comment is less likely to be disapproved of. 

\subsubsection{Fixed Effects}
A summary of our fixed effects is given by our research questions.

\subsubsection{Random Effects}
We grouped intercepts by Author and by Subreddit. As Reddit is pseudononymous, some authors may have a reputation for making consistently good or bad posts, and different communities may have different standards for acceptability.
\section{Results}
As we have twenty models and are thus making twenty comparisons, we will use the Bonferoni correction to avoid false positives \cite{bonferoni}. This means we'll use $p=0.0025$ as a threshold for significance.

\subsection{Positive Comments}
For positive comments made, we see a reliable effect due to fatalities, separatist conflicts, and terrorist conflicts. As a reminder, more fatalities increases the likelihood of a comment being considered unacceptable in this case, while Terrorist and Separatist conflicts decrease that likelihood. See Table~\ref{posfix}.

\begin{table}
\centering
\begin{tabular}{l|llll}
& Estimate & Std. Error & z value & Pr$(<|z|)$  \\ \hline
Intensity.L & 0.3561 & 0.6668 & 0.534 & 0.593 \\ 
Fatalities & 0.1091 &  $<0.001$ & 118 & $<0.00001$* \\ 
IDP & 0.3545 & 0.2568 & 1.38 & 0.167 \\ ne
Refugees & 0.1361 & 0.3371 & 0.404 & $0.686$ \\ 
Separatist & -0.5505 & $<0.001$ & -632 & $<0.00001$* \\ 
Criminal & 0.5866 & 0.5738 & 1.022 & 0.307 \\ 
Ethnic & -2.4295 & 3.2196 & -0.755 & 0.45 \\ 
Terrorism & -0.8033 & $<0.001$ & -861 &  $<0.00001$* \\ 
Territorial & -1.133 & 1.279 & -0.885 & 0.376 \\ 
Foreign & 1.0048 & 0.5995 & 1.676 & 0.0938. \\ 
Marginal Fatalities & 0.3844 & 0.2611 & 1.472 & 0.141 \\ 
Marginal Refugees & 0.2577 & 0.3425 & 0.752 & 0.452 \\ 
Marginal IDP & 0.3223 & 0.2868 & 1.124 & 0.261 \\ 
Age & -0.2801 & 0.3285 & -0.853 & 0.394 \\ 
Region-EastAsia/Australia & -2.486 & 2.998 & -0.829 & 0.406894 \\ 
Region-Europe  & -0.4437 &  0.827 & -0.536 & 0.591615 \\ 
Region-MiddleEast/NorthAfrica & -0.2767 & 0.8544 & -0.324 & 0.74601 \\ 
Region-Russia/Eurasia & -1.191 & 1.643 & -0.725 & 0.468473 \\ 
Region-SouthAsia & -2173 & >99999 & 0.00 & 0.999955 \\ 
Region-SubsaharanAfrica & 2.097 & 2.300 & 0.912 & 0.361987 \\ 
\end{tabular}
\caption{The positive sentiment models. Region was fitted as a factor. Intensity was fitted as an ordinal where only the linear is reported. Significant variables are marked with an asterisk.}
\label{posfix}
\end{table}

\subsection{Negative Comments}
For negative comments made, we see reliable effects in the case of Territorial conflicts, age, and for the regions of Europe and East Asia / Australasia. The regions and age all make the comments less likely to be downvoted, and the conflict being territorial makes the comment more likely to be downvoted. See Table~\ref{negfix}.

\begin{table}
\centering
\begin{tabular}{l|llll}
& Estimate & Std. Error & z value & Pr$(<|z|)$  \\ \hline
Intensity.L & 0.15252 & 0.06973 & 2.187 & 0.0287 \\ 
Fatalities & 0.01045 &  0.02465 & 0.424 & $0.672$ \\ 
IDP & 0.01904 & 0.03315 & 0.574 & 0.566 \\ 
Refugees & -0.06791 & 0.03972 & -1.71 & $0.0873$ \\ 
Separatist & -0.10355 & 0.05655 & -1.831 & 0.0671 \\ 
Criminal & 0.09881 & 0.05588 & 1.768 & 0.077 \\ 
Ethnic & -0.25600 & 0.16045 & -1.595 & 0.111 \\ 
Terrorism & 0.12486 & 0.05634 & 2.216 &  $0.0267$ \\ 
Territorial & 0.45897 & 0.07979 & 5.753 & $<0.00001$* \\ 
Foreign & 0.09581 & 0.06056 & 1.582 & 0.114 \\ 
Marginal Fatalities & 0.009692 & 0.030874 & 0.314 & 0.754 \\ 
Marginal Refugees & -0.06146 & 0.04131 & -1.488 & 0.137 \\ 
Marginal IDP & -0.05012 & 0.03845 & -1.303 & 0.192 \\ 
Age & -0.11456 & 0.02767 & -4.14 & $<0.0001$* \\ 
Region-EastAsia/Australasia & -0.52475 & 0.15148 & -3.250 & 0.00116* \\ 
Region-Europe  & -0.4437 &  0.827 & -0.536 & $<0.00001$* \\ 
Region-MiddleEast/NorthAfrica & 0.13825 & 0.07504 & 1.842 & 0.06542 \\ 
Region-Russia/Eurasia & -0.18945 & 0.08913 & -2.125 & 0.03355 \\ 
Region-SouthAsia &  & >99999 & 0.00 & 0.999955 \\ 
Region-SubsaharanAfrica & 0.17248 & 0.15783 & 1.093 & 0.27448 \\ 
\end{tabular}
\caption{The negative sentiment models. Region was fitted as a factor. Intensity was fitted as an ordinal where only the linear is reported. Significant variables are marked with an asterisk}
\label{negfix}
\end{table}
\section{Discussion}
There are several ways to interpret the rate of unacceptable posts for any given conflict. As the majority of Reddit users are fairly homogeneous in opinion, it is possible that people vary how likely they are to make posts about certain topics when they know that many people will disapprove. This variance could stem from \textit{passion} about a topic: someone feels it is more important to speak their mind than to have a popular post. Alternatively, it could reflect \textit{dissent}; while there might not be enough dissenters to affect the gatekeeping system, the willingness to express something others disapprove of at all suggests there are some who disagree with the majority opinion. 

\subsection{Negative Sentiment}
Due to the vast majority of comments being of negative sentiment, we assume it is the default way to respond to a conflict. Thus, we will consider the Negative Sentiment model the same as the Default model.

The Regions of East Asia/Australasia and Europe both have reliable negative predictors. This implies the rate of disapproval when discussing these topics is very low. This is somewhat unsurprising given the general demographics of Reddit. Users from those regions, due to comparable socioeconomic status or military alliances, may see themselves as similar, prompting a homophily effect. The lack of significant effects for the other regions could be because commenters conflate areas with which they are less likely to empathize, such as the Middle East/North Africa and SubSaharan Africa.

Age having a negative effect makes sense from the perspective of passion. While many people may have an opinion on older conflicts, these feelings may be less immediate due to the numbness and weariness of prolonged violence.

Territorial disputes on the other hand, are very logically controversial. To those whose country plays a role or are immediately affected by them, they may seem very important. However, to those farther away, they could seem like petty bickering.

Interestingly, the objective variables, such as Fatalities and IDP, played almost no role in the model. Absence of evidence is not evidence of absence; however, this integrates well with previous work \cite{Berinsky2009}.

\subsection{Positive Sentiment}
We can assume positive sentiment corresponds to hope, optimism, or perhaps even sarcastically phrased misanthropic sentiments. For instance, in the case of comments about conflicts with higher Fatalities being more likely to be disapproved of, it might be the latter. 

On the other hand, positive sentiment surrounding conflicts containing notes of Terrorism or Separatism may correspond to hope. For instance, they could be expressions of hope for those attempting to separate from a regime where they do not feel represented, or wishes for those who are suffering terrorism to remain steadfast. These should both be uncontroversial ideas, so it is unsurprising they are more likely to be considered acceptable.
%One possible interpretation of unacceptable posts Reddit users to post in anger. If Reddit is fairly homogenous in viewpoint, as a lack of controversial comments would suggest, positive correlations could reflect someone who posted something they knew would be disagreed with. In this way, someone who found the Reddit explanation of events frustrating could become angry and post a message that is likely to be downvoted. In this way, it would be sensible for Age to be negatively correlated, as fewer people are as passionate about historical events as they are about present day events.

\subsection{Future Work}
Some of the interpretation is ultimately speculative. A more fine-grained model of sentiment could potentially help us determine with more authority what is the true cause of these effects. We would further like to tie these ideas into more general cognitive and perceptual biases. For instance, one possibility of the low effect of negative externalities could be due to poor estimates of those values. Lastly, these results are only really trying to describe how people in the West talk about conflicts that are mostly elsewhere. It would be interesting to demonstrate the effect of a distance metric that was independent of where the commenter was. Other features, such as the nature of the conflict, are likely to be context-dependent and are thus difficult to measure in a locale-independent way.
\section*{Acknowledgements}
This research was supported by the National Science Foundation under grants titled ``Updating the Militarized Dispute Data Through Crowdsourcing'' (SBE-SES-1528624) and ``Alignment in webforum discourse'' (CISE-IIS-1459300).
\section{Conclusion}
In conclusion, we were able to perform exploratory data analysis on a large sample of data to examine how users of the Reddit Internet forum talk about armed conflicts. There seem to be many features that affect these conversations, including the location the conflict took place in, the age of the conflict, whether a foreign power was a significant antagonist in the conflict, and whether the conflict was separatist in nature. The actual externalities were not significant, which could imply that subjective views of conflicts are more important to the discussion than the objective data.  

\bibliographystyle{splncs03}
\bibliography{review,nlp,reddit}


\end{document}
