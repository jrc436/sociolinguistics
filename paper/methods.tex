\section{Methods}
\subsection{Data Preparation}
Interestingly, the sentiment analysis provided nearly twenty times more negative comments than positive comments. It is possible that this is due to the subject matter; in general, pity and sadness might be more common responses to tragedy than optimism and hope. It is also possible that some positive comments actually reflected misanthropic views about violence. Due to the difficulty of interpreting neutral sentiment comments, they were excluded.

Due to Reddit's manipulation of the probability of any given user seeing a post, and because controversy is so rare in our data set, we did not consider the actual number of upvotes or downvotes a comment received to be useful information. We thus coded anything upvoted at all as acceptable, and anything downvoted at all as unacceptable. We excluded comments that received no votes.

Our two filters resulted in 781 positive comments and 14289 negative comments. If the Positive Sentiment Model is substantially different than the Negative Sentiment Model, that suggests sentiment, perhaps through interaction with other variables, does influence acceptability. It is possible negative comments can be interpreted as the norm, instead of as a special case. If this is true, then that model, instead of reflecting when negative sentiment would be unacceptable, would reflect in general when posters are more likely to post something that is unacceptable. 

\begin{table}
\centering
\begin{tabular}{l|rrrr}
& Estimate                    & Std. Error & z value & Pr$(<|z|)$  \\ \hline
Intensity                   & 0.3561 & 0.6668 & 0.5340 & 0.59300 \\ 
Fatalities                    & 0.1091 &  $<0.001$ & 118.0000 & $<0.00001$* \\ 
IDP                           & 0.3545 & 0.2568 & 1.3800 & 0.16700 \\ 
Refugees                      & 0.1361 & 0.3371 & 0.4040 & 0.68600 \\ 
Separatism                    & -0.5505 & $<0.001$ & -632.0000 & $<0.00001$* \\ 
Criminal Violence             & 0.5866 & 0.5738 & 1.0220 & 0.30700 \\ 
Ethnic Violence               & -2.4295 & 3.2196 & -0.7550 & 0.45000 \\ Terrorism                     & -0.8033 & $<0.001$ & -861.0000 &  $<0.00001$* \\ 
Territorial Dispute           & -1.1330 & 1.2790 & -0.8850 & 0.37600 \\ 
Foreign Antagonism            & 1.0048 & 0.5995 & 1.6760 & 0.09380 \\ 
Marginal Fatalities           & 0.3844 & 0.2611 & 1.4720 & 0.14100 \\ 
Marginal Refugees             & 0.2577 & 0.3425 & 0.7520 & 0.45200 \\ 
Marginal IDP                  & 0.3223 & 0.2868 & 1.1240 & 0.261000 \\ 
Age                           & -0.2801 & 0.3285 & -0.8530 & 0.39400 \\ 
Region-EastAsia/Australasia     & -2.4860 & 2.9980 & -0.8290 & 0.40689 \\ 
Region-Europe                 & -0.4437 &  0.8270 & -0.5360 & 0.59162 \\ 
Region-MiddleEast/NorthAfrica & -0.2767 & 0.8544 & -0.3240 & 0.74601 \\ 
Region-Russia/Eurasia         & -1.1910 & 1.6430 & -0.7250 & 0.46847 \\ 
Region-SouthAsia              & -2173.0000 & $>99999.0000$ & 0.0000 & 0.99996 \\ 
Region-SubsaharanAfrica       & 2.0970 & 2.3000 & 0.9120 & 0.36199  
\end{tabular}
\caption{The \textbf{positive} sentiment models. Region was fitted as a factor. Intensity was fitted as an ordinal variable where only the linear effect is reported. Significant variables are marked with an asterisk.}
\label{posfix}
\end{table}

\subsection{Model}
We use a Logistic Linear Mixed Effects Model (GLMM) to attempt to explain which sentiments were acceptable as determined by the conflict. Due to missing data for certain features (such as the marginal Refugees for any given year) and a high number of predictors, we decided to fit individual models for each predictor. This allows us to also see precisely how well each variable explains the variance. However, as we are making more than one comparison, we have to adjust our test of significance to avoid false positives. We use the Bonferoni correction \cite{bonferoni}, resulting in a significance threshold of $p=0.0025$.

As every comment is coded as either unacceptable (1) or acceptable (0), positive $\beta$ relates to downvotes and negative $\beta$ relates to upvotes, which correspond with disapproval and approval effectively.

We grouped intercepts by Author and by Subreddit. As Reddit is pseudononymous, some authors may have a reputation for making consistently good or bad posts, and different communities may have different standards for acceptability.