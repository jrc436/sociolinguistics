\section{Discussion}
We can potentially divide our variables into two categories: objective variables, and variables that cause bias. Objective variables in this case include things like the number of people who died in the conflict. Regardless of where someone is from, that number means the same thing. Bias variables could vary based on who is reading it. While Reddit has users from all over the world of all ages, ethnicities, genders, and so on, at least half are from the United States \footnote{https://reddit.zendesk.com/hc/en-us/articles/205183225-Audience-and-Demographics}. Reddit users also generally tend to be younger, rather than older \footnote{http://www.pewinternet.org/2013/07/03/6-of-online-adults-are-reddit-users/}. As the majority of Reddit is in English, it can be assumed that the site is fairly culturally homogeneous. Thus, the views of Reddit should be partially reflective of the views of the young Western world.

Considering this, it is perhaps unsurprising that the region Europe had a very significant negative effect. This implies that people are unlikely to make comments that are disapproved of about conflict in Europe. As East Asia and Australasia are lumped together, it is possible that that part of the world is seen as synonymous with the Western World. It's also possible there is more empathy for East Asia for other reasons, such as Reddit's passion for Asian culture (citation needed). The lack of significant effects for the other regions could be due to a sort of "othering", where places that seem less relevant are lumped together.

It is also unsurprising that conflicts with foreign involvement would be more likely to be disapproved of. These comments could potentially occur due to massive disdain for foreign invasions among Reddit's liberal audience (citation needed). Separatist conflicts being unlikely to generate a negative response could be due to empathy, due to America's own separatist history.

Interestingly, the objective variables played almost no role in any model ($p > 0.5$), while Intensity did play a role. While Intensity and Fatalities are significantly correlated, it's possible that Intensity also, to some extent, reflects public perception better than any of the purely numeric data. 

One possible interpretation of negative vs. positive effects stems from the tendency of Reddit users to post in anger. As Reddit can be fairly homogenous, positive correlations could actually reflect disagreement. In this way, someone who found the Reddit explanation of events frustrating could become angry and post a message that is likely to be downvoted. In this way, it would be sensible for Age to be negatively correlated, as fewer people are as passionate about historical events as they are about present day events.