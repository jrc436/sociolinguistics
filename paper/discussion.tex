\section{Discussion}
Our variables fall into two categories: objective variables, and variables whose interpretation is subject to cultural bias. Objective variables in this case include measure such as the number of people that died in the conflict. Regardless of cultural factors, this number is objective. Culturally biased variables may vary with the reader's cultural stance. While Reddit has users from all over the world of all ages, ethnicities, genders, and so on, at least half are from the United States\cite{reddit-demo}. Reddit users also generally tend to be younger, rather than older \cite{pewinternet}. The website is fairly culturally homogeneous anglosaxon, with the top four countries by unique visitors in 2013 were the United States, Canada, the United Kingdom, and Australia, followed by Germany and Japan\footnote{Source: http://www.redditblog.com/2013/12/top-posts-of-2013-stats-and-snoo-years.html}. Thus, the views of Reddit should be partially reflective of the views of the young Western world as well as East Asia.  We do not claim this to be a representative sample of the world population.  In fact, Region was a strong predictor of polarized sentiment acceptability is likely to be correlated with proximity of the conflict.

Considering this, it is perhaps unsurprising that the region Europe had a very significant negative effect. This implies that people are unlikely to like or make comments that disapprove of conflict in Europe. As East Asia and Australasia aggregated, it is possible that due to comparable socioeconomic status or military alliances, these parts of the world are seen as more similar to the West, prompting a homophily effect. The lack of significant effects for the other regions could be caused by conflating regions that are more difficult to feel empathy for.

It is also unsurprising that conflicts with foreign involvement would be more likely to be disapproved of. These comments could potentially occur due to young people's skepticism of foreign invasions in a post-Iraq world \cite{cato}. 

Americans may also feel empathetic in the case of separatist conflicts, making negative responses less likely.

Interestingly, the objective variables played almost no role in any model ($p > 0.5$), while Intensity did play a role. While Intensity and Fatalities are significantly correlated, it's possible that Intensity also, to some extent, reflects public perception better than any of the purely numeric data. 

One possible interpretation of unacceptable posts  Reddit users to post in anger. If Reddit is fairly homogenous in viewpoint, as a lack of controversial comments would suggest, positive correlations could reflect someone who posted something they knew would be disagreed with. In this way, someone who found the Reddit explanation of events frustrating could become angry and post a message that is likely to be downvoted. In this way, it would be sensible for Age to be negatively correlated, as fewer people are as passionate about historical events as they are about present day events.