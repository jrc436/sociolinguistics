\section{Introduction}
According to Armed Conflict Database, there are 42 active conflicts around the world causing 180,000 fatalities and more than 12 million refugees. For those who live in relatively peaceful area, such as North America, their perception, opinion and action towards these international crises are important. Collectively, these perceptions are influential in shaping their countries' foreign policy \cite{Gelpi2009}. 

Traditional media and social media interact to help form these perceptions. Conversely, consumers of media implicitly or explicitly rate the quality of the content. Thus, contributors' perception of events forms something of an exacerbatory cycle, where biases can potentially further biases. The question is what are those biases to begin with?

We are motivated to understand how people talk about and act upon armed conflicts. Specifically, we want to understand when and why people accept negative discussions and positive discussions. We examine the effect of features like the number of fatalities, refugees and Internally Displaced People (IDP),  along with a variety of features that may influence objectivity, such as the location of the conflict. We hope to discover relations between these features and public opinions of specific armed conflicts.