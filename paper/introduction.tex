\section{Introduction}
According to Armed Conflict Database, there are 42 active conflicts around the world causing 180,000 fatalities and more than 12 million refugees. For people who live in a relatively peaceful area, like north American, their perception, opinion and action towards these international crises are important as collectively they are influential in shaping..... 

We are motivated to understand how people talk about and act upon armed conflicts. More specifically, we want to understand when and why people accept negative discussions and positive discussions, and what roles the features of conflicts play in the discussions.

Researches on the opinions of armed conflicts occuring around the world were found from the perspectives of both decision makers and public. Cognitive consistency, distortion, dissonance and inertia, and their implications in perceiving international politics were examined with polititians \cite{Jervis1976}. Public opinion on armed conflicts from World War II to Vietnam, the Gulf, Afganistan and Iraq have also been investigated in order to find patterns of public responses to international conflicts. Berinsky has found that public reactions to conflicts has been shaped less by their defining characteristics, such as fatalities and resources costs, than by the same political interests and group affiliations that influence ideas about domestic issues \cite{Berinsky2009}. Particularly, with the pervasive access of the Internet and use of social media, public opinions are becoming more influential \cite{Shirky2011}. In addition, social media is becoming prominent agents/platforms to frame views and coordinate actions on certain conflict events, for example, arranging protests and uprising \cite{Lim2012}.