\section{Related Study}
\subsection{Public Opinion of Armed Conflicts}
In general, there are not so many researches targeting public perceptions in armed conflicts. Studies that are investigating cognitive consistency, distortion, and dissonance and inertia, and their implications in the area of international politics are aiming to understand how decision makers perceive armed conflicts \cite{Jervis1976}.

With the pervasive use of social media, the influence of public opinion is becoming more influential, from Vietnam war to Iraq war to refugee crisis recently peaked with Syria civil war. Also, social media is becoming prominent agents/platforms to act upon certain conflict events, for example, arranging protests and uprising (citation Merlyna Lim, 2012).

Social media has been found to have effects over military and government restriction of information, newsroom dynamics and audience engagement in reportage of conflicts and war (citation sacco and Bossio 2015), media framing of conflicts do affect public's tolerance thought altering the perceived importance of public order values (citation nelson, clawson, oxley, 1997).  In addition, even though communication technologies make it possible to gather and deliver every conflict in the world, government interest, political priorities, and news organizations' view of the world brings bias and limitations for news audience of international conflicts. Let alone the fair to poor performance and shrinkage of international coverage of American news agencies make it harder for public to get broad-spectrum of information they need to evaluate governments’ performance (citation seib philip 2004).


\subsection{Similar Studies}
The usage of large-scale datasets from social media have produced a flood of researches and discussions in areas like sociology, politics, psychology and so on. The roles of social media as an object in coordinating activities pre-conflict, during conflict and post-conflict have been studied extensively. For example, time-series analysis of big social media datasets offer an evolving theories of public opinion and public attention on economic and social welfare, foerign affairs, and environmental issues. 


\section{Database Description}
\subsection{Armed Conflict Database}
From the Armed Conflicts Database developed by International Institute for Strategic Study (IISS), we obtain various indexes of armed conflicts around the world. As we are interested in recent armed conflicts, we select the ones that were happening at least in one of the three years from 2012 to 2014. In total, we collected 48 conflicts covering regions including Caribbean and the Americas, East Asia and Australasia, Europe, Middle East and North Africa, Russia and Eurasia, South Asia, and Sub-Saharan Africa. In general, there are five types of conflicts from our dataset: criminal violence, ethnic conflict, separatist conflict, territorial conflict, foreign involvement and terrorism. Majority of the conflicts are inherently a combination of those types. For example, Xinjiang conflict in China is categorized as ethnic conflict, separatist conflict, and also terrorism. All the armed conflicts have relatively different life span, intensity, as well as number of casualties, refugees, and Internally Displaced People (IDP). The average years of these armed conflicts worldwide is about 23 years.

\subsection{Reddit Comment Dataset}
