\section{Related Work}
\subsection{Public Opinion of Armed Conflicts}
We investigated past research on the perception of armed conflicts from the perspectives of both the decision makers and the general public. Cognitive consistency, distortion, dissonance and inertia, [[THIS IS SURE A LOT OF TERMS THAT DON'T NECESSARILY MEAN ANYTHING TO ANYONE]] and their implications in perceiving international politics were examined with decision makers \cite{Jervis1976}. Public opinion on armed conflicts from World War II to Vietnam, the Gulf, Afganistan and Iraq war have also been investigated to find patterns of public responses to international conflicts. Particularly, with the pervasive access of the Internet and use of social media, public opinions are becoming more influential \cite{Shirky2011}. [[MORE INFLUENTIAL HOW, ON THE POLITICAL CYCLE, POLITICAL DECISION-MAKING, OR WHAT]] However, systematic studies of public opinion from social media on armed conflicts is limited. Studies focusing on social media in armed conflicts often regard social media tools as agents or platforms to frame views and coordinate actions: for example, arranging protests and coordinating uprisings \cite{Lim2012}.

Public perception of armed conflicts through [[SOCIAL? TRADITIONAL?]] media suffers from several limitations. Advanced communication technologies make it possible to disseminate information about various conflicts around the world \cite{Sacco2015}. However, government interests and news organizations' will frame  conflicts in a certain way that adds their own bias. Even if it is unintentional, this causes a significant limitation for the public to be able to obtain the broad-spectrum of information they need to evaluate such conflicts objectively \cite{Nelson1997}. Furthermore, news agencies generally focus on news in their own country or nearby countries, shrinking the amount of time focused on international coverage \cite{Seib2004}. 

Despite these biases, growing studies are covering the relationship between the characteristics of armed conflicts and the public perception of them. Baum looked at how entertainment-oriented, soft news expose politically inattentive individuals to information about high-profile foreign crises \cite{Baum2002}. [[WHY IS THIS SIGNIFICANT? HOW DOES THIS TIE INTO OUR FEATURES]] Berinsky has found that public reactions to conflicts has been shaped less by their defining characteristics, such as fatalities and resources costs, than by the same political interests and group affiliations that influence ideas about domestic issues [[WHAT DOES THIS ACTUALLY MEAN? IS THERE A GOOD EXAMPLE OF IT?]] \cite{Berinsky2009}. Gartner at el. investigated this idea more deeply; he found that marginal fatalities are only important in explaining opinion when fatality accumulation is accelerating \cite{Gartner1998}.[[IN OTHER WORDS, SHOULD RECENT FATALITIES OUTWEIGH TOTAL FATALITIES?]] They also studied the relationship between race and the sensitivity towards fatalities in war \cite{Gartner2000}. [[WHAT DID THEY FIND? ARE AFRICAN AMERICANS MORE SENSITIVE TO FATALITIES IN AFRICA?]]

\subsection{Social Computing}
The usage of large-scale datasets from social media have produced a flood of research and discourse in areas like sociology, political science, and psychology. The roles of social media as a platform in coordinating activities pre-conflict, during the conflict and post-conflict have been studied extensively. For example, time-series analysis of big social media datasets offers evolving theories of public opinion and public attention on economic and social welfare, foreign affairs, and environmental issues \cite{RussellNeuman2014}. Twitter, in the realm of social media, is one of the most popular corpora for studying armed conflicts and other political events. 


Few studies have examined public opinions of multiple armed conflicts together by using social media data, especially the relaitonship between charateristics of the conflicts and the level of acceptability of public opinions.
