\section{Background}
\subsection{Public Perception of Armed Conflicts}
Public opinion on armed conflicts from World War II to Vietnam, the Gulf, Afganistan and Iraq war have also been investigated to find patterns of public responses to international conflicts \cite{Berinsky2009}. However, with prevalent access to the Internet and social media, public opinions are becoming more influential in political decision-making; they are even changing the political cycle \cite{Shirky2011}. However, systematic studies of public opinion from social media on armed conflicts is limited. Studies focusing on social media in armed conflicts often regard social media tools as agents or platforms to express views and coordinate actions: for example, arranging protests and organizing uprisings \cite{Lim2012}.

Public perception of armed conflicts through both traditional and social media suffers from several biases. Advanced communication technologies make it possible to disseminate information about various conflicts around the world \cite{Sacco2015}. However, government and news organizations cover and frame conflicts in a certain way that adds their own bias. For example, when covering foreign events, reporters largely focus on war, terrorism and political violence \cite{Nossek2004}. Even still, most of the world's conflicts are largely unreported by the media \cite{Hawkins2011}. Even if it is unintentional, this causes a significant limitation for the public to be able to obtain the broad-spectrum of information they need to evaluate conflicts objectively \cite{Nelson1997}. News agencies in general focus on news in their target audience's country or nearby countries, shrinking the amount of time focused on international coverage \cite{Seib2004}.

Despite these biases, a growing number of studies are covering the relationship between the characteristics of armed conflicts and the public perception of them. Berinsky has found that public reactions to conflicts have been shaped less by their defining characteristics, such as fatalities and resource costs, than by one's political affiliations \cite{Berinsky2009}. For instance, if someone's political party supports a war, then she most likely does as well. Furthermore, Gartner et al found that marginal fatalities, which are the number of fatalities that occurred that year, are more important in explaining opinion than the cumulative number of fatalities \cite{Gartner1998}. They also studied the relationship between race and opinion towards the Vietnam war, finding that people were likely to view the conflict more negatively based on the number of people who died that were in the same locale, regardless of race \cite{Gartner2000}. Surprisingly, they found that Asian Americans had greater support for the Vietnam War; this could be because some of them were fleeing communism themselves \cite{Gartner2000}. 

\subsection{Social Computing}
The usage of large-scale datasets from social media has produced a flood of research and discourse in areas like sociology, political science, and psychology. Researchers have extensively studied the role of social media as a platform in coordinating activities before, during, and after conflicts. For example, time-series analysis of social media datasets has offered an ever-evolving account of public opinion and attention on a variety of issues, such as economic and social welfare, foreign affairs, and environmental issues \cite{RussellNeuman2014}. While studies have looked at public perception of individual armed conflicts, we are not aware of any studies that have leveraged the mass of social media data to examine what causes perceptual differences among them.
%Twitter, in the realm of social media, is very popular for studying social phenomena \cite{twitter1} \cite{twitter2} \cite{twitter3}. 