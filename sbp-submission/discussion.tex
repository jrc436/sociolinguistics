\section{Discussion}
There are several ways to interpret the rate of unacceptable posts for any given conflict. As the majority of Reddit users are fairly homogeneous in opinion, it is possible that people vary how likely they are to make posts about certain topics when they know that many people will disapprove. This could stem from \textit{passion} about a topic: someone feels it is more important to speak their mind than to have a popular post. Alternatively, it could reflect \textit{dissent}; while there might not be enough dissenters to affect the gatekeeping system, the willingness to express something others disapprove of at all suggests there are some who disagree with the majority opinion. 

\subsection{Negative Sentiment}
Due to the vast majority of comments being of negative sentiment, we assume it is the default way to respond to a conflict. Thus, we will consider the Negative Sentiment model the same as the Default model.

The Regions of East Asia/Australasia and Europe both have reliable negative predictors. This implies the rate of disapproval when discussing these topics is very low. This is somewhat unsurprising given the general demographics of Reddit. Users from those regions, due to comparable socioeconomic status or military alliances, may see themselves as similar, prompting a homophily effect. The lack of significant effects for the other regions could be because commenters conflate areas with which they are less likely to empathize, such as the Middle East/North Africa and Sub-Saharan Africa.

The negative effect of Age makes sense from the perspective of passion. While many people may have an opinion on older conflicts, these feelings may be less immediate due to the numbness or weariness of prolonged violence.

Territorial disputes on the other hand, are logically controversial. To those whose country plays a role or are immediately affected by them, they may seem existential. However, to those farther away, they could seem like petty bickering.

Interestingly, the objective variables, such as Fatalities and IDP, played almost no role in the model. Absence of evidence is not evidence of absence; however, this integrates well with previous work \cite{Berinsky2009}, which likewise found social connections to be more predictive than fatalities or cost of a conflict.

\subsection{Positive Sentiment}
We can assume positive sentiment corresponds to hope, optimism, or perhaps even sarcastically phrased misanthropic sentiments. For instance, in the case of comments about conflicts with higher Fatalities being more likely to be disapproved of, it might be the latter. 

On the other hand, positive sentiment surrounding conflicts containing notes of Terrorism or Separatism may correspond to hope. For instance, they could be expressions of hope for those attempting to separate from a regime where they do not feel represented, or wishes for those who are suffering terrorism to remain steadfast. These should both be uncontroversial ideas, so it is unsurprising they are more likely to be considered acceptable.
%One possible interpretation of unacceptable posts Reddit users to post in anger. If Reddit is fairly homogenous in viewpoint, as a lack of controversial comments would suggest, positive correlations could reflect someone who posted something they knew would be disagreed with. In this way, someone who found the Reddit explanation of events frustrating could become angry and post a message that is likely to be downvoted. In this way, it would be sensible for Age to be negatively correlated, as fewer people are as passionate about historical events as they are about present day events.

\subsection{Future Work}
Some of the interpretation is ultimately speculative. A more fine-grained model of sentiment could potentially help us determine with more authority what is the true cause of these effects. We would further like to tie these ideas into more general cognitive and perceptual biases. For instance, one possibility of the low effect of negative externalities could be due to poor estimates of those values. Lastly, it would be interesting to see how these results generalize to communities besides Reddit.