We then analyze comments based on two main features: \textit{sentiment} and \textit{acceptability}. We rated over 25,000 comments using the StanfordNLP Sentiment Analyzer \cite{stanfordnlp}. The acceptability refers to how the community perceives the comment and its sentiment. In this case, an acceptable comment is one that the community would give upvotes, and an unacceptable comment is one the community would give downvotes.

We wanted to examine several characterizing features of conflicts that affect how they are discussed by traditional news agencies.

\begin{enumerate}
\item{\textbf{Severity.} This includes the total number of people who were killed, made refugees, or internally displaced as a result of the conflict. While common sense would suggest that severity should play a large role in perception, prior work suggests it plays a very modest role in public perception of conflicts.} 
\item {\textbf{Region.} There are six major regions that the Armed Conflict Database groups conflicts into. We hypothesize that regions that share more cultural similarity would be viewed differently than regions which share less. More specifically, regions where Reddit users are common would be perceived differently than regions where they are not.} 
\item {\textbf{Marginal Severity.} This includes the number of people who were killed, made refugees, or internally displaced the same year as the comment was made. From previous research, we suspected that marginal severity would play a larger role than total severity.}
\item {\textbf{Age.} The number of years that have passed since the conflict started. We hypothesized age could be significant due to waning interest. After decades of conflict, it is perhaps difficult for some users to still empathize with ongoing tragedy.}
\item {\textbf{Nature.} This includes a set of attributes: Separatism, Terrorism, Foreign Antagonism, Territorial Disputes, Criminal Violence, and Ethnic Violence. We hypothesized that attributes that are easier for Westerners to empathize with due to history, such as Separatism or Foreign Antagonism, might be treated differently. Further, attributes that seem very foreign or 'uncivilized', such as Ethnic Violence or Territorial Disputes, may also be seen differently. Importantly, a conflict can have more than one of these attributes.}
\item {\textbf{Expert Perception.} The Armed Conflict Database rates conflicts for their current level of intensity. While this is obviously correlated with deaths, refugees, and IDP ($p < 0.0001$), the average person may be more likely to get their ideas of severity from experts, rather than numbers. This could also influence the selection process and tone of traditional media articles.} 
\end{enumerate}