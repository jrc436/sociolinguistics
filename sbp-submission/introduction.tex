\section{Introduction}
According to the Armed Conflict Database, there are 42 active conflicts around the world causing 180,000 fatalities and more than 12 million refugees. For people who live in relatively peaceful areas, such as in North America, their perception, opinion and action towards these international crises are important. Collectively, these perceptions are influential in shaping their countries' foreign policy \cite{Gelpi2009}. 

Traditional media and social media interact to help form these perceptions. On the other hand, consumers of media implicitly or explicitly rate the quality of the content produced. Contributors and evaluators naturally have biases that can build on each other, if contributions that follow these biases are rated more positively than those that do not. The question is what are those biases to begin with?

In particular, we seek to answer this question in the context of armed conflicts. For instance, how do people talk, evaluate, and contribute to the discussion of various armed conflicts? Specifically, we want to understand when and why people accept negative discussions and positive discussions. We examine the effect of features such as the number of fatalities, refugees and Internally Displaced People (IDP). Further, we investigate a variety of characteristics that may influence objectivity, such as the location of the conflict and the number of years the conflict were active. We hope to discover the relationship between these features and the perception of specific armed conflicts.

To accomplish this, we turn to the Reddit Internet forum. Reddit's relative homogeneity in both demographics and opinion give it certain advantages over other social media. As will be explored later, much of Reddit consists primarily of young Western people. Further, we will also demonstrate how their opinions on these armed conflicts are fairly homogeneous. This gives us an ideal dataset to reflect on a specific culture's judgments of acceptability.

%Reddit's homogenaeity
%Tie to Journalism principles